\documentclass[11pt,a4paper]{article}
\usepackage[utf8]{inputenc}
\usepackage{amsmath, amssymb, amsthm}
\usepackage{hyperref}
\usepackage{graphicx}

\title{Fixed-Point Congruences in Base-16 Kaprekar Transforms and Heuristic Decay in Collatz Orbits}
\author{Enrique Coello-montoya \\ \small Computational Research Division}
\date{February 2, 2026}

\begin{document}

\maketitle

\begin{abstract}
This paper documents the computational identification of the hexadecimal Kaprekar sequence for $n < 10^6$. We further investigate the structural stability of the Collatz $3n+1$ conjecture by quantifying the parity drift, identifying a consistent entropy bias of $\approx 0.68$. These results suggest that hexadecimal digit-sum transforms follow predictable modular congruences.
\end{abstract}

\section{Introduction}
The Kaprekar property, traditionally studied in decimal (Base-10), defines a set of integers $n$ such that $n^2$ can be partitioned into two substrings whose sum equals $n$. We extend this search to hexadecimal (Base-16) to observe the impact of increased radix on sequence density.

\section{Methodology: Hexadecimal Search}
Using the SageMath environment, we executed an exhaustive search across the interval $[0, 10^6]$. The search algorithm identifies $n$ such that:
$$n^2 \equiv n \pmod{16^k - 1}$$
where $k$ represents the partition index.

\subsection{Key Findings}
The search yielded several non-trivial anomalies, most notably the high-magnitude integer \textbf{953,250}. The full sequence $S$ is provided in the associated repository.

\section{Collatz Parity Drift}
The $3n+1$ conjecture is often treated as a pseudo-random process. However, by calculating the geometric mean of multipliers $M = \{0.5, \approx 3.0\}$ over $10^4$ iterations, we observe a "Gravity Constant" $G$:
$$G = \exp\left( \frac{1}{N} \sum_{i=1}^{N} \ln(m_i) \right) \approx 0.68$$
This indicates a structural bias toward the $4-2-1$ attractor, as $G < 1.0$ necessitates long-term orbital decay.

\section{Conclusion}
The identification of the Base-16 Kaprekar set provides a new dataset for OEIS indexing. Future work will focus on the asymptotic density of these numbers as $n \to \infty$.

\end{document}
